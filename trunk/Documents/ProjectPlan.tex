\documentclass[onside,10pt,left=1cm]{article}
\oddsidemargin 0in
\evensidemargin 1in
\topmargin 0in
\textheight 9in
\textwidth 6in

\author{Christer Vindberg \and Kim Birkelund}
\title{DoVMfOOL - project plan}

\usepackage{listings}
\usepackage{graphicx}

\lstset{language=c,breaklines=true,prebreak={\space},frame=single,numbers=left}

\newcommand{\csharp}{C\#}
\newcommand{\dotnet}{.NET}
\newcommand{\cpp}{C++}
\newcommand{\java}{Java}
\newcommand{\jit}{JIT}
\newcommand{\gbeta}{gbeta}
\newcommand{\vm}{VM}
\newcommand{\api}{API}

\newcommand{\bold}[1]{$\mathbf{#1}$}

\begin{document}

\maketitle

\pagenumbering{arabic}
\setcounter{secnumdepth}{2}

The purpose of our project is not so much solving a problem as it is
simply implementing a virtual machine. Seeing as how we have no
experience in this field it seemed like the obvious choice. Likewise
since we have no experience in creating large projects in \cpp~we have
chosen to do the implementation in \csharp/\dotnet. Obviously this is not
without problems, both performance wise and technical, since we don't
have direct memory access. We feel however that the amount of time
we would spent trying to make \cpp~do what we want it better spent on
designing the actual \vm.

The \vm itself will be executing a dynamic class based language. We
have opted to take a (almost) pure message-passing approach. Each
class can define any number of message-handlers along with a default
handler. Each class can also define fields but these are only
accessible from inside the class - all fields are, in \java~terms,
declared as private. 

Memory management wise hope to be able to implement a generational
garbage collector, but will probably start with a simple copying
collector - which might simply evolve into our generation zero
collector. In terms of execution we will start out with an interpretor
but hope to be able to implement an \jit-compiler utilizing the built-in
emitting \api of the \dotnet~framework.

With all this in mind we have defined the following milestones:

\begin{itemize}

\item \bold{M1}: Can execute byte-code language. This will not
  necessarily include accessibility enforcement or
  multiple-inheritance resolution.

\item \bold{M2}: Full language executable. In addition to the
  omissions in M1 this should include an assembler from text to a
  direct loadable binary format.

\item \bold{M3}: Garbage collection. Simple copy-collector.

\item \bold{M4}: Generational collector.

\item \bold{M5}: \jit-compiler. May not include inline-caching.

\item \bold{M6}: \dotnet~interoperability, threading,
  inline-caching, trace-based compilation, compiling from \gbeta.

\end{itemize}

The schedule is currently to allocate one milestone for each week
starting with this one (week 44). This means that we should be able to
execute our first programs at the end of this week and going foreword
satisfying each milestone at the end of the corresponding week. The
features described in milestones M4 to M6 are to be seen as
nice-to-have and not must-have. This means that should an earlier
milestone slip we are prepared to sacrifice those features.

\end{document}

%%% Local Variables: 
%%% mode: latex
%%% TeX-master: t
%%% End: 
